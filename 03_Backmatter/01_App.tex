\chapter{Response to the referees}
\section{Doc. PhDr. Adam Ger\v{s}l Ph.D.}

    \textit{Comments to the first paper on ``What Type of Finance Matters for Growth? Bayesian Model Averaging Evidence''}

    \begin{enumerate}
       
    \item \textit{The title is confusing - the paper does not explore what type of finance (bank versus market; bond versus stocks; banks versus non-bank institutions; short-term versus long-term; concentrated versus unconcentrated banking sector etc.) matters for growth, but what aspect of financial intermediation (financial depth; activity on markets; efficiency of banks; resilience of banks) matter. I propose to adjust the title accordingly.}
    
    Response 1 continues here and goes along.

    \item \textit{The endogeneity problem (briefly mentioned on p. 11) is much more serious than the author thinks as the variables get averaged over 50 years! The studies focusing on the dynamics of development between the real and financial sector (macrofinancial linkages, such as in Crowe et al. 2010) emphasize the two-way interactions and feedbacks that develop over time. As the methodology does not take into account the time dynamics, the thesis should at least acknowledge that endogeneity could be an issue and devote a paragraph or so to this shortcoming, adding a few references on the (omitted) dynamic interactions.}
    
    \item \textit{Throughout the paper, the net interest margin (NIM) is interpreted as a measure of ``efficiency of financial intermediaries'', but this is incorrect! It is a measure of (in)efficiencies in financial intermediation, not an indicator of bank (cost) efficiency! Large NIMs are typical for underdeveloped markets in which risks (of default), vulnerabilities, and legal uncertainties (of collateral realization etc.) are large, increasing information asymmetries and creating frictions to (efficient) financial intermediation. Thus, the NIM is actually an additional (indirect) measure of the institutional (legal) framework within which financial intermediation takes place rather than ``an aspect'' of financial activity.}

    \item \textit{Given the previous point, the author should be much more careful in drawing conclusions from the analysis. The fact that his measure of efficiency has a large PIP might be to a large extent related to the endogeneity bias: as an economy develops, the overall risks and vulnerabilities decline, contributing to a decline of the margin, which goes hand in hand with expansion in lending, further supporting economic development. This link should be mentioned in the paper.}    
    
    \item \textit{The review of literature mentions the criticism of traditional finance-growth nexus papers in neglecting the private bond markets (and other non-bank or non--stock--market sources of finance), but the paper again uses only the two traditional measures of financial depth -- bank credit and stock market (p. 10). Could the private credit to GDP be based on the BIS statistics of total credit (i.e. a sum of bank credit, non-bank intermediaries credit, bonds issues, and cross-border finance to private sector)? This has become available recently for a large number of countries (and years) and could better capture the debt of the private sector intermediated by all intermediaries and markets.}
    
    \textit{Comments to the second paper on ``Finance and Wealth Inequality''}

    \item \textit{In comparison to the previous paper, this one tackles well a possible endogeneity bias (section 3.5.3) by lagging the explanatory variables (average of 1980-2009) compared to the dependent variable (average of 2010-2016), although I would be less concerned about the reverse link between inequality and financial development (compared to GDP growth and financial development). While a paragraph explaining through which channels would wealth inequality influence financial development is included on p. 66 (with reference to Beck et al. 2007), I am not too persuaded by (the two) explanations. Could 1-2 additional references be provided if, as the author states, ``the question of endogeneity is deeply ingrained in the finance-inequality nexus''? (Some additional arguments are included in the third paper and could be re-used here).}
    
    \item \textit{In this paper, an overall index of efficiency is used, combining the net interest margin with variables such as overhead costs or profitability. I would still propose that this index is not called Financial Institutions Efficiency (but perhaps Financial Intermediation Efficiency) because it combines institutions' (cost) efficiency and efficiency (frictions) of financial intermediation influenced by overall risks. As this indicator has a 100\% PIP, it would be worth exploring further what drives the efficiency index -- is it more the NIM/spread as a measure of inefficiencies in financial intermediation (risks) or (overhead) costs as a measure of financial institutions' efficiency?}
    
    \textit{Comments to the third paper on ``Finance and Inequality – Panel BMA Approach''}

    \item \textit{The paper could better formulate what is the value added compared to available literature. There are a few hints in the last paragraph on page 99, but this could be better structured and start with ``The value added of this paper is in\dots''. Using the after-tax rather than before-tax measure of income distribution should also be mentioned here (and explained why).}
    
    \item \textit{Averaging the data across 3-year spans to produce the panel (over 2000-2014) is a standard method to get rid of short-term volatility in the data say from markets, but given that this coincides with the largest economic pre-crisis boom 2000-2007 and the crisis and post-crisis decline associated with balance-sheet recessions in many countries (2008/2009-2014), it will not get rid of the business cycle. The more traditional 5-year averaging would be better. As this was done as a robustness check (with similar results), I would propose to use the 5-year averaging as a baseline and the 3-year averaging as a robustness check.}
    
    \item \textit{Another robustness check could be to split the sample into two periods only, pre-crisis 2000-2007 and post-crisis 2008-2014, creating two 7-year averages. Instead of a panel, two cross-sections would be run and results could be interpreted as regime-specific (pre-crisis versus post-crisis regimes).}
    
    \item \textit{Is endogeneity a potential problem here (similarly to the second paper) and how is it dealt with?}
\end{enumerate}