\chapter{Response to the referees}
\section{Doc. PhDr. Adam Ger\v{s}l Ph.D.}

\textit{Comments to the first paper on ``What Type of Finance Matters for Growth? Bayesian Model Averaging Evidence''}

% {\itshape The dissertation thesis is a collection of three very well written empirical papers exploring the links between financial development on one hand and long-term economic growth, wealth inequality, and income inequality on the other hand. The author demonstrates strong skills in using up-to-date econometric methods, especially the Bayesian Model Averaging (BMA), to analyze available data on relevant economic phenomena. The first paper titled ``What Type of Finance Matters for Growth? Bayesian Model Averaging Evidence'' explores the finance-growth link with an updated dataset on long-term per capita GDP growth, a number of economic and institutional factors, and selected variables capturing financial intermediation. The value added is in using the BMA framework to account for model uncertainty and in testing the explanatory power of five selected financial development indicators: in addition to the traditional credit to GDP and stock market capitalization (both capturing the depth of financial intermediation), the paper uses a measure of bank efficiency (net interest margin), a measure of stock market activity (turnover rate), and a measure of bank stability (the Z-score). The author finds that especially the bank efficiency measure is robustly related to long-term economic growth, which much higher inclusion probability than the traditional financial depth indicators.

% The second paper on ``Finance and Wealth Inequality'' uses the BMA framework to explore the link between financial, economic and institutional factors and the wealth Gini coefficient for a sample of 73 countries. The results show that finance has a multi-faced impact on wealth inequality – financial depth increases inequality, but efficiency and access to finance decrease it.

% The final paper investigates the link between financial intermediation and income inequality within a panel BMA framework. Also this paper finds that finance has a complex impact on inequality – the access to finance and efficiency decrease it, while the size of markets does not have any significant impact.

% There is an original contribution of the author in all three essays, all are based on relevant references, and all would be in my view defendable as a part of dissertation at respected universities. They are all written in a clear language and definitely publishable (actually, the first one has been published already in the World Bank Economic Review and the second in the Journal of International Money and Finance). While the focus is on data analysis and deriving relationships based on regression techniques, the thesis would benefit from a few revisions that would emphasize economic understanding, especially when interpreting the findings of the statistical exercises.

% My general assessment is that the thesis can be defended after revision indicated in my comments below.}

\begin{enumerate}
       
    \item \textit{The title is confusing - the paper does not explore what type of finance (bank versus market; bond versus stocks; banks versus non-bank institutions; short-term versus long-term; concentrated versus unconcentrated banking sector etc.) matters for growth, but what aspect of financial intermediation (financial depth; activity on markets; efficiency of banks; resilience of banks) matter. I propose to adjust the title accordingly.}
    
    Thank you for the valid point and a suggestion of an alternative. I changed the title of the first paper to \emph{What Aspect of Financial Intermediation Matters for Growth: Bayesian Model Averaging Evidence}. I adjusted the footnote referring to the publication of the chapter so that is reflects the title of the published paper.

    \item \textit{The endogeneity problem (briefly mentioned on p. 11) is much more serious than the author thinks as the variables get averaged over 50 years! The studies focusing on the dynamics of development between the real and financial sector (macrofinancial linkages, such as in Crowe et al. 2010) emphasize the two-way interactions and feedbacks that develop over time. As the methodology does not take into account the time dynamics, the thesis should at least acknowledge that endogeneity could be an issue and devote a paragraph or so to this shortcoming, adding a few references on the (omitted) dynamic interactions.}
    
    \item \textit{Throughout the paper, the net interest margin (NIM) is interpreted as a measure of ``efficiency of financial intermediaries'', but this is incorrect! It is a measure of (in)efficiencies in financial intermediation, not an indicator of bank (cost) efficiency! Large NIMs are typical for underdeveloped markets in which risks (of default), vulnerabilities, and legal uncertainties (of collateral realization etc.) are large, increasing information asymmetries and creating frictions to (efficient) financial intermediation. Thus, the NIM is actually an additional (indirect) measure of the institutional (legal) framework within which financial intermediation takes place rather than ``an aspect'' of financial activity.}

    \item \textit{Given the previous point, the author should be much more careful in drawing conclusions from the analysis. The fact that his measure of efficiency has a large PIP might be to a large extent related to the endogeneity bias: as an economy develops, the overall risks and vulnerabilities decline, contributing to a decline of the margin, which goes hand in hand with expansion in lending, further supporting economic development. This link should be mentioned in the paper.}    
    
    I have extended space devoted to the discussion of endogeneity. Although in a crude way, I tackle it in robustness check with lagged values of financial indicators. The time span is shorter (10 years) and admittedly involves aftermath of financial crisis, but nevertheless, the conclusions are consistent with the baseline.

    \item \textit{The review of literature mentions the criticism of traditional finance-growth nexus papers in neglecting the private bond markets (and other non-bank or non--stock--market sources of finance), but the paper again uses only the two traditional measures of financial depth -- bank credit and stock market (p. 10). Could the private credit to GDP be based on the BIS statistics of total credit (i.e. a sum of bank credit, non-bank intermediaries credit, bonds issues, and cross-border finance to private sector)? This has become available recently for a large number of countries (and years) and could better capture the debt of the private sector intermediated by all intermediaries and markets.}
    
    \textit{Comments to the second paper on ``Finance and Wealth Inequality''}

    \item \textit{In comparison to the previous paper, this one tackles well a possible endogeneity bias (section 3.5.3) by lagging the explanatory variables (average of 1980-2009) compared to the dependent variable (average of 2010-2016), although I would be less concerned about the reverse link between inequality and financial development (compared to GDP growth and financial development). While a paragraph explaining through which channels would wealth inequality influence financial development is included on p. 66 (with reference to Beck et al. 2007), I am not too persuaded by (the two) explanations. Could 1-2 additional references be provided if, as the author states, ``the question of endogeneity is deeply ingrained in the finance-inequality nexus''? (Some additional arguments are included in the third paper and could be re-used here).}
    
    \item \textit{In this paper, an overall index of efficiency is used, combining the net interest margin with variables such as overhead costs or profitability. I would still propose that this index is not called Financial Institutions Efficiency (but perhaps Financial Intermediation Efficiency) because it combines institutions' (cost) efficiency and efficiency (frictions) of financial intermediation influenced by overall risks. As this indicator has a 100\% PIP, it would be worth exploring further what drives the efficiency index -- is it more the NIM/spread as a measure of inefficiencies in financial intermediation (risks) or (overhead) costs as a measure of financial institutions' efficiency?}
    
    \textit{Comments to the third paper on ``Finance and Inequality – Panel BMA Approach''}

    \item \textit{The paper could better formulate what is the value added compared to available literature. There are a few hints in the last paragraph on page 99, but this could be better structured and start with ``The value added of this paper is in\dots''. Using the after-tax rather than before-tax measure of income distribution should also be mentioned here (and explained why).}
    
    \item \textit{Averaging the data across 3-year spans to produce the panel (over 2000-2014) is a standard method to get rid of short-term volatility in the data say from markets, but given that this coincides with the largest economic pre-crisis boom 2000-2007 and the crisis and post-crisis decline associated with balance-sheet recessions in many countries (2008/2009-2014), it will not get rid of the business cycle. The more traditional 5-year averaging would be better. As this was done as a robustness check (with similar results), I would propose to use the 5-year averaging as a baseline and the 3-year averaging as a robustness check.}
    
    \item \textit{Another robustness check could be to split the sample into two periods only, pre-crisis 2000-2007 and post-crisis 2008-2014, creating two 7-year averages. Instead of a panel, two cross-sections would be run and results could be interpreted as regime-specific (pre-crisis versus post-crisis regimes).}
    
    \item \textit{Is endogeneity a potential problem here (similarly to the second paper) and how is it dealt with?}
\end{enumerate}

\section{Prof. Dr. Ansgar H. Belke}
% {\itshape My comments on the above mentioned dissertation which is in the fields of research on financial development are as follows. 

% The idea behind this paper is nice, to take up-to date (Bayesian) econometric techniques and to apply them empirically to the issue of the relation between finance and growth and finance and inequality, respectively. The problem and topic of the individual papers, i.e. the chapters of the dissertation, is clearly set out in the abstract. And I also think – as will become obvious in the following – that the paper meets its own aims formulated in the introduction. 

% The dissertation deserves its main merits for its empirical efforts based on up-to-date econometric methods. The way in which the author implements his econometric strategies in the specific cases does appear overall adequate. 

% He has already published two papers contained in the dissertation in excellent refereed journals, i.e. the World Bank Economic Review and the Journal of International Money and Finance. These two papers form the first two chapters in the dissertation. He has written his third paper on the effect of financial development on income inequality by himself. It is very promising in terms of publication success as well. Most importantly, I can clearly recognize an original contribution of the author who has done all the excellent econometrics on his own.

% In his dissertation, the author proves his ability to grasp the main concepts, paraphrase them, and apply them. He is able to reflect on the main concept underlying the relevant theories and their interpretations. Seen on the whole, he proves to be intellectually independent. And there is sustained and direct engagement with the main research question. Hence, it does not come as a surprise that the author obviously widely understands the implications of his research questions and comes up with persuading answers to these research questions. 

% Concerning the structure of the argument it can be stated that the author has chosen a coherent and cogent structure which proceeds logically from point to point and paragraph to paragraph. There is a sustained train of thought throughout. Finally, it appears fair to say that the use of illustrating examples and tables and figures is judicious and highly appropriate. Moreover, there is a good balance between factual detail and key theoretical issues. 

% A large share of the relevant literature has been identified as well. The thesis is thus based on relevant references. 

% The thesis would also be defendable at my home institution or another respected institution where I gave lectures such as the Main University of Vienna (Austria) or the University of Hohenheim (Germany).

% In the following, I would like to deal with some minor issues which could be dealt with by the candidate in the upcoming defense:}

\begin{enumerate}
    \item \textit{A main issue to cope with in at least two of the papers (if not all) is endogeneity. The author should explain the strengths and weaknesses of the solutions found and applied in his papers to deal with this issue as, for instance, the use of lagged regressors etc. Are there some tasks and open issues left for future research in that respect?}
    
    \item \textit{A linear functional form which is implicitly often assumed in the literature is fairly specific and, in some cases, even restrictive. It is important to distinguish specifications which can be examined in the framework of a linear regression from those which cannot. It is nice that the author thus checked for functional form beforehand and also implemented and estimated non-linear specifications. The author could comment a bit more on the chosen tests for non-linearity.}
    
    \item \textit{What about (further) robustness checks? Does the author exploit all usual possibilities to conduct robustness checks (changes of the lag structure, explicit parameter restriction tests, preliminary sample split tests according to different policy regimes also beyond the financial crisis, changes of the criteria which serve as the basis for selecting the final presented empirical models such as information criteria) in the framework of his analysis? If not, please complement or at least be more explicit on what has been done.}

    \item \textit{What about (further) robustness checks? Does the author exploit all usual possibilities to conduct robustness checks (changes of the lag structure, explicit parameter restriction tests, preliminary sample split tests according to different policy regimes also beyond the financial crisis, changes of the criteria which serve as the basis for selecting the final presented empirical models such as information criteria) in the framework of his analysis? If not, please complement or at least be more explicit on what has been done.}

    \item \textit{At certain stages of his dissertation, the author applies cross-sectional data analysis. The author should be explicit about why he is not using panel data at these stages of analysis and what the trade-offs and sacrifices of this way of proceeding are.}
    
    \item \textit{So, is there any relevance of the paper for policy issues beyond that briefly and partly implicitly mentioned in the conclusions? I would appreciate if the authors would not only come up with testable hypotheses and the respective empirical results using readily available data but bring the very useful discussion of why and how finance matters for growth and inequality closer to the realm that is applicable to policymakers.}

    \item \textit{However, the summary of the dissertation is missing which should have given an overview of the dissertation and the research questions tackled therein. In this sense, it would have been quite useful as a guide for the reader.}
\end{enumerate}

\section{Martin \v{C}ih\'{a}k Ph.D.}

\textbf{a) Contribution}
Combined, the essays compiled in this thesis offer useful and original contributions to the considerable and expanding empirical literature on the intersection of finance, growth, and inequality. On a personal note, I have appreciated the ingenious use of the GFDD database that I spearheaded when I was at the World Bank. I find that this type of rigorous empirical approach can truly improve our understanding of the role of finance in the economy.

I have three comments/suggestions for clarifications on the thesis’ contribution:

\begin{enumerate}
    \item The document unfortunately appears to be incomplete, because Chapter 1 is missing/blank. The thesis would benefit from a well-crafted introductory chapter.
    \item Given that chapters 2 and 3 are both joint with two coauthors (Mr. Mares being listed as the last of the three), it may be useful to clarify Mr. Mares’s contribution. I presume that Mr. Mares has contributed significantly, but there is no way for me to ascertain the precise extent of Mr. Mares’s involvement. It would be helpful for the thesis to contain an upfront disclosure/statement about the nature and scope of Mr. Mares’s contribution to each of the co-authored essays (perhaps on the same page as the ``declaration of authorship''), indicating what are the contributions of Mr. Mares, and what are those of his coauthors.
    \item Chapter 4 seems an extension of chapter 3, using the same Bayesian Model Averaging (BMA) approach to the finance-inequality nexus, the difference being that chapter 4 looks at income instead of wealth proxies. If there are other notable differences or contributions, it may be useful to flag any novel contributions upfront (perhaps in the forthcoming chapter 1).
\end{enumerate}

% \textbf{b) References}

% I have found the papers to be well-researched and generally well-motivated by gaps in the existing empirical literature. I provide some suggestions for references below.

% The thesis is relatively parsimonious, perhaps to the point of being telegraphic, on the theoretical front. The author concentrates on taking the empirical technique—BMA—to the data on finance and inequality. I do have sympathy for this heavily empirical approach. Nonetheless, for better balance, some more theoretical/conceptual discussion on the finance-inequality nexus would be useful.
% One of the key issues in the financial sector of the last decade has been the rise of shadow banking, and more recently the ascent of ``FinTech''. The thesis is focusing on banks, and for good reason. Nonetheless, it could come across as out-of-touch with the above developments, so I suggest to at least mention the rise of fintech and its potential effects, including on inequality.
% The thesis includes many relevant references, but the list is far from comprehensive, so perhaps that can be flagged upfront. Also, as part of strengthening the policy discussion, the author could reference the following relevant IMF Staff Discussion Notes (SDN):

% \begin{itemize}
% \item \href{https://www.imf.org/en/Publications/Staff-Discussion-Notes/Issues/2020/01/16/Finance-and-Inequality-45129}{SDN 20/01 ``Finance and Inequality''}
% \item \href{https://www.imf.org/en/Publications/Staff-Discussion-Notes/Issues/2018/09/17/women-in-finance-a-case-for-closing-gaps-45136}{SDN 18/05 ``Women in Finance: A Case for Closing Gaps''}
% \item \href{https://www.imf.org/en/Publications/Staff-Discussion-Notes/Issues/2016/12/31/Financial-Inclusion-Can-it-Meet-Multiple-Macroeconomic-Goals-43163}{SDN 15/17 ``Financial Inclusion: Can it Meet Multiple Macroeconomic Goals?''}
% \item \href{https://www.imf.org/en/Publications/Staff-Discussion-Notes/Issues/2016/12/31/Rethinking-Financial-Deepening-Stability-and-Growth-in-Emerging-Markets-42868}{SDN 15/08 ``Rethinking Financial Deepening: Stability and Growth in Emerging Markets''}
% \end{itemize}

% \textbf{c) Suitability for defense}
% The thesis has many strong aspects that certainly make it defendable. However, at my home institution, the IMF, I would expect the study to include a clearer discussion of policy implications, and an overall summary. In that context, it is a pity that chapter 1 (``Summary of the dissertation'') is blank in the current draft. For the next/revised draft, I would suggest including chapter 1, which should ideally pull together the various storylines from the three separate essays and weave them into one coherent whole.

% \textbf{d) Suitability for publication}

% There is no doubt in my mind that much of the thesis is publishable in respected journals. I have enjoyed reading the essays, especially chapters 2 and 3.
% Chapter 1 is missing, so it is hard to pass judgement, but chapter 2 has already been published in the World Bank Economic Review. Moreover, chapter 3—already issued as an IES working paper—is forthcoming in Journal of International Money and Finance, so it is certainly publishable. Chapter 4 appears relatively less ripe than the other two essays, but even that chapter can be suitable for publication. With chapter 4, readers may wonder about its contribution relative to chapter 3. Both chapters feature a heavy use of Bayesian Model Averaging (BMA) to study the link between finance and inequality. A difference is that chapter 3 focuses on wealth proxies while chapter 4 looks as income proxies, but some readers may be left wondering about the relative value added of chapter 4 and whether the two could perhaps be combined.

\textbf{e) Comments}
\begin{enumerate}
    \item \textit{In chapter 2, the finding that quality of finance matters is intellectually appealing, but I would caution that the measure of net interest margin is only a partial proxy for ``efficiency''. In particular, net interest margin captures factors such as asset composition of financial intermediaries. To truly evaluate efficiency of financial intermediaries, one needs to look also at other measures, such as cost-to-income ratios or overhead costs to total assets (which are also in the GFDD). Following up on the earlier general point, having a solid conceptual/theoretical discussion of ``efficiency of financial intermediaries'' may be helpful before diving into the BMA and using it on the net interest margin as a proxy.}
    
    \item \textit{In chapter 2, the discussion on nonlinearity (section 2.5.3) comes across as an afterthought. Given the massive attention in the recent literature on nonlinearities in the relationship between finance and growth, it is surprising to see this aspect to receive only a relatively scant attention. Nonlinearity in the finance-growth nexus (and finance-inequality nexus) may well be a part of the reason why linear relationships (examined in much of this paper) can come out insignificant.}
    
    \item \textit{In chapters 3 and 4, it would be helpful to clarify the different concepts of inequality, how they are measured, and how they relate to each other. It is important to flag that the precision of some commonly used inequality indicators has become a subject of major public controversy and discussion (Auten and Splinter 2019; Bhalla 2017; Economist 2019) In light of these important debates, a fuller discussion of weaknesses of existing inequality measures seams important. Please consider adding (and discussing) the following references:}
     \begin{itemize}
        \item Auten, Gerald, and David Splinter. 2019. ``Top 1 Percent Income Shares: Comparing Estimates Using Tax Data.'' AEA Papers and Proceedings 109:307–11.
        \item Bhalla, Surjit S. 2017. The New Wealth of Nations. Simon and Schuster
        \item Economist. 2019. ``Inequality Illusions: Why Wealth and Income Gaps Are Not What They Appear.'' November 30
     \end{itemize}

    \item \textit{Chapter 3: given the importance of instruments for the BMA estimation, please consider a more specific discussion of the instruments. For example, why is the average of areas 3D, 4C, 4D, and 5A of the EFW a suitable instrument? The authors claim that ''components of our financial liberalization measure are exogenous to the wealth inequality as the change in wealth distribution is improbably to have direct effect on any of them``. It is unclear where this assertion comes from, so if there is evidence for it, I suggest adding it. There is some evidence to the contrary, at least for income inequality (see for example Sylwester, Kevin, 2010, Journal of Applied Economics), finding strong evidence of links between inequality and the black market premium (which is one of the EFW areas, namely 4C).}
    
    \item \textit{Chapter 4 departs from much of existing literature by considering the after-tax rather than the before-tax income distribution as a dependent variable. However, the rationale for this choice is not well explained. In fact, one could make the case for considering before-tax income distribution, because that’s the one where financial sector’s role is likely to be more prominent/visible (and more separable from the effects of other policies, including fiscal).}
    
    \item \textit{The thesis—across the chapters—would benefit from strengthening the discussion on policy implications. For example, chapter 2 says that ``the current wave of regulatory changes intended to safeguard financial stability should carefully analyze the consequences for the efficiency of financial intermediaries,'' which would benefit from clarification. Are the authors suggesting to reorient micro- and macro-prudential supervisors from safeguarding financial stability to targeting efficiency of financial intermediaries? (I presume not, but the text could be misread that way.) Also, the reference to ``the current wave of regulatory changes'' seems outdated and misleading, given that the post-crisis regulatory wave has already taken place (and we are now in the stage where some countries are considering regulatory roll-backs.)}

    \item \textit{Chapter 2, page 20: the regression includes various dummy variables, such as the one for Sub-Saharan Africa and the ``fraction of Confucian population'' (which is close to a proxy for China). Given the importance of these regions, I worry that the estimated coefficients on those dummy variables are just proxies for our ignorance about the underlying drivers of the finance-growth relationship. It would be useful to include a solid discussion on these dummy variables.}

    \item \textit{Chapter 2, front page footnote: ``the The World Bank Econonmic Review'' should read ``The World Bank Economic Review''.}
\end{enumerate} 