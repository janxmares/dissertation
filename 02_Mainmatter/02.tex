\chapter{What Type of Finance Matters for Growth? Bayesian Model Averaging Evidence}
\label{chap:two}
\blfootnote{This chapter was co-authored with Iftekhar Hasan and Roman Horv\'{a}th and published in the \emph{The World Bank Econonmic Review}. We thank Martin Feldkircher and seminar participants at 19th ICMAIF conference (Rethymno, Greece), 32nd International Symposium on Money, Banking and Finance (Nice, France) and 1st World Congress for Comparative Economics (Rome, Italy) for helpful comments. The views expressed here are those of the authors and not necessarily those of the Czech Ministry of Finance or Bank of Finland.}

\begin{quote}
\begin{center}\textbf{Abstract}\end{center}
	We examine the effect of finance on long-term economic growth using Bayesian model averaging to address model uncertainty in cross-country growth regressions. The literature largely focuses on financial indicators that assess the financial depth of banks and stock markets. We examine these indicators jointly with newly developed indicators that assess the stability and efficiency of financial markets. Once we subject the finance-growth regressions to model uncertainty, our results suggest that commonly used indicators of financial development are not robustly related to long-term growth. However, the findings from our global sample indicate that one newly developed indicator -- the efficiency of financial intermediaries -- is robustly related to long-term growth.     
	\end{quote}

\clearpage
\section{Introduction}
\label{ch2sec:intro}
Numerous studies investigate the effect of financial development on economic growth and predominantly conclude that there is a positive causal relationship between the two \citep{KingLevine1993a, LevineZervos1998, AtjeJovanovich1993}. Nevertheless, some opposing views hold that the financial sector removes scarce resources from the rest of the economy \citep{Tobin1984,Boltonetal2011} and encourages to greater exposure and vulnerability to crises, thus severely burdening the real sector during periods of instability \citep{Kindelberger1978,Minsky1991,Stiglitz2000}. The effect of financial development on growth has recently drawn greater attention again because of the financial crisis that began in 2007-2008. Moreover, conclusions referring to diminishing and eventually negative returns from financial development have become increasingly common in the literature \citep{Arcandetal2012,CecchettiKharroubi2012,LawSingh2014}. This highlights the importance of the financial sector for the functioning of the economy and has provoked extensive debate among policymakers. 

This paper evaluates the finance-growth nexus but differs from previous research in two main respects. First, it employs Bayesian model averaging (BMA) to overcome certain drawbacks of previous research approaches. BMA is well grounded in statistical theory \citep{Rafteryetal1997} and addresses the inherent regression model uncertainty, which is quite high in cross-country growth regressions \citep{Fernandezetal2001, SalaiMartinetal2004, Durlaufetal2008}. The control variables in finance-growth regressions are often selected in a somewhat \textit{ad hoc} manner with reference to certain relevant theories while ignoring other relevant theories. 

BMA essentially allows us to control for dozens of potentially relevant determinants of growth within a unifying framework. The variety of theories of economic growth has given rise to a large number of determinants and resulted in substantial uncertainty concerning the true growth model. In essence, the BMA procedure estimates different combinations of explanatory variables and subsequently weights the coefficients using various measures of model fit. As a consequence, BMA also conveniently limits concerns regarding omitted variable bias and its adverse consequences of inconsistently estimated coefficients, an issue that is typically abstracted from in the empirical work on finance and growth. BMA is capable of evaluating numerous possible regressors and estimating their posterior inclusion probability (PIP), i.e., the probability that they are relevant in explaining the dependent variable, in addition to the weighted mean and variance of their corresponding coefficients. While model averaging has become standard in the empirical growth literature \citep{SalaiMartinetal2004, Durlaufetal2008}, it has not been applied to study the finance--growth nexus.  

Second, we differ from previous research by examining additional financial indicators to appreciate the multidimensionality of financial systems. Importantly, previous research, including recent studies implying that excessive financial development harms growth \citep{Arcandetal2012,CecchettiKharroubi2012,LawSingh2014}, largely focuses on measures of the depth of financial development such as the credit to GDP ratio. We depart from existing literature in jointly examining whether the depth, stability or efficiency of financial markets (or all of them) is crucial for long-term growth. In doing so, we can unify and re-examine previous studies on the finance-growth nexus that show that a) financial development is conducive to growth, b) excessive financial development is not, and c) financial instability has negative consequences for growth.

The theoretical concepts regarding the functions of the financial industry are difficult to operationalize in empirical research, and there is no universal consensus regarding the measurement of financial development \citep{KingLevine1993a}. Although measuring financial development is complex, researchers typically consider only those variables capturing financial depth, such as the credit to GDP ratio or stock market capitalization, to assess the degree of financial development. Financial indicators assessing the degree of financial access, financial stability or the efficiency of the financial industry have largely been ignored in cross-country studies due to data limitations. The newly developed Global Financial Development Database (GFDD) represents a significant improvement in this respect and provides a comprehensive set of financial indicators that reflect various functions and characteristics of the financial sector. In addition to financial depth, the GFDD provides measures of the efficiency and stability of and access to financial markets. Although data availability remains somewhat limited, we extend the existing literature by including these additional dimensions of the financial sector in our regression analysis to more completely evaluate the effect of finance on growth. Specifically, the indicators we use represent the depth, stability, and efficiency of the banking sector and stock markets as defined by \citet{Cihaketal2013}.\footnote{We did not include financial access indicators because of data unavailability. In our sample, the data on the proxy variable recommended for financial access by \citet{Cihaketal2013}, bank accounts per 1,000 adults, are missing for 36 out of 60 countries. Including financial access in the analysis would therefore severely limit our cross-section of countries resulting in the non-negligible loss in the degrees of freedom. Nevertheless, we have examined alternative (less than ideal) financial access indicators from the GFDD database (bank branches per 100,000 adults and ATMs per 100,000 adults), which are available for almost all countries in our sample. However, we fail to find these indicators to be decisive for the long-term growth.} In addition to the GFDD, we employ the widely used dataset on the determinants of long-term growth developed by \citet{Fernandezetal2001}, which encompasses over 40 explanatory variables capturing various economic, political, geographical, and institutional indicators. 

While it is commonly assumed that causality goes from financial development to economic growth, some scholars argue that a growing financial sector merely follows the increasing needs of the real economy or may be determined simultaneously with growth due to other factors. The quantitative survey of the finance and growth literature by \citet{Valickovaetal2014}, for example, indicates that those studies ignoring endogeneity are more likely to report a stronger positive effect of financial development on growth. Although it is likely that a part of endogeneity in finance--growth nexus can be addressed by model averaging procedure (reducing omitted variable bias), we also examine the robustness of our results through specifications that employ the lagged explanatory variables. To the best of our knowledge, this is the first study to combine various characteristics of the financial sector, a rich dataset on growth, and an approach that addresses model uncertainty and endogeneity. As a result, our study addresses two main issues in finance--growth literature: 1) causality issues and 2) measurement of financial development.

Using data on real economic growth in 60 countries between 1960 and 2011, we find that bank efficiency is robustly related to long-term growth and exhibits very high PIP. This finding corresponds to the predictions of theoretical model by \citet{Pagano1993}, who shows that the efficiency of financial intermediaries is crucial for funneling savings to investment and therefore, for increasing real growth. The relevance of traditional variables, such as credit provided to the private sector or stock market capitalization, is weaker. In addition, we also fail to find a non-linear effect of financial development on growth.  Our results are robust to a series of checks such as employing a different sample period, different parameter priors or addressing endogeneity. Therefore, our results highlight that the approach to measuring financial development is crucial for the estimated effect of finance on growth. Our policy implication is that those managing the current worldwide wave of regulatory changes in the financial industry should not underestimate the importance of the efficiency of financial intermediaries for long-term growth.

This paper is structured as follows. Section 2 provides a literature review on finance and growth. Section 3 presents the data. We describe Bayesian model averaging in section 4. We provide the regression results in section 5. The conclusions are presented in section 6. An appendix with additional results follows. 






