\begin{refsection}
\chapter{Summary of the dissertation}
\label{ch1}
The Great Recession following the financial crisis of 2007-2008 reinvigorated the interest in the research of financial development and its impact on the real economy. Although it was a dominant view that more finance is good for economic growth and equalizes opportunities, the crisis spurred questions about the non-linearities in finance effects and the existence of a healthy threshold or the impact being conditional on the quality of institutions. New regulatory waves followed, focusing primarily on the stability of the individual financial intermediaries and the overall systemic risk. 

The measurement of finance and financial development also came into question. Most of the researchers in the field relied on proxies capturing the size (depth) of finance. However, these are imperfect proxies of the functions ascribed to finance in the theoretical models \parencite{Levine2005}. When I was about to begin my dissertation research, more detailed and sufficiently dense data on financial development were published by the World Bank. The information on the stability, efficiency, and access to financial intermediaries was not perfect, but it indicated better alternatives to size in accurately capturing the channels through which finance affects the real economic phenomena. Together with the carefully adopted methodological approach in BMA, which allows for a comparison of the relative importance of different financial proxies, it suggested a promising research path I decided to take. Economic growth was a straightforward first choice for the initial scientific paper, given the relevance of the question at the time. We prototyped the applicability of our approach and put forward novel evidence about the finance-growth nexus. I simultaneously followed the evolution of new and updated inequality measures as the Capital in the 21st Century \parencite{piketty2014} became one of the best-sellers in the decade. The literature on the inequality determinants seemed as ambiguous as the one on growth. Applying the proven toolkit to strenuously collected data on wealth inequality promised an entirely novel contribution to the literature. Together with co-authors, we uncovered particular aspects of finance as essential determinants of wealth inequality across countries. The paper became the second piece of research presented in this dissertation. A further spin-off to income inequality where we explore a potentially heterogeneous effect of financial development on income distribution then naturally followed to constitute the final chapter of the dissertation. Altogether, the dissertation is a composition of three papers related to financial development and its consequences. 

By financial development, I refer to developing financial contracts, markets, and intermediaries that facilitate the screening of investment opportunities, the monitoring of investments, and the pooling, trading, and management of risk. I thus adopt Tobin's functional view of finance, which also mentions the facilitation of transactions by providing mechanisms and networks of payments, reflecting the financial industry's economic value \parencite{Tobin1984}. Looking at the specific proxies of the functions mentioned above, I intend to explore how financial development increases social welfare rather than particular market agents' business efficiency. In other words, I take macroeconomic rather than a microeconomic exploratory path\footnote{It is essential to note that I deliberately switch between the efficiency of financial intermediaries and financial intermediation efficiency, always having in mind the presently stated functions and macroeconomic view of finance.}.  It contrasts the dissertation from other branches of literature that very specifically look at the profitability and technical efficiency of financial intermediaries, such as stochastic frontier analysis in \textcite{bonin2005privatization} or \textcite{cici2018trading} for stock market trading efficiency.

The financial development measures we employ are still indirect, albeit providing a closer approximation of functions ascribed to finance in economic theory.  In the latter chapters, they become more sophisticated and multi-faceted by combining several underlying indicators related to the same function of finance. In their classification, we rely on \textcite{Cihaketal2013} and \textcite{svirydzenka2016introducing} who pioneered the field. A repetitive pattern arises in the presented papers. We confirm the imperfection of typical size indicators of financial development and simultaneously identify access to finance and financial intermediation efficiency as informative in explaining differences in economic growth, wealth, and income inequality.

In selected chapters that follow, I often refer to the authorship as ``we'', which reflects that some of the papers in this dissertation I wrote in collaboration with my supervisor professor Roman Horv\'{a}th and professor Iftekhar Hasan from Fordham University. If I have to self-evaluate, my contribution to these papers was substantial in all research stages, from drafting the ideas, data collection, analysis, drafting the paper, and responding to referees during the publication process. I continue with the overviews of individual dissertation papers.

In \emph{Chapter 2 - What Type of Finance Matters for Growth? Bayesian Model Averaging Evidence}, we examine the effect of finance on long-term economic growth. We consider the size proxies jointly with indicators that assess the stability and efficiency of financial markets. In the paper, we address the inconclusive finance-growth nexus literature. While some claim financial development has positive effect on economic growth \parencite{AtjeJovanovich1993,KingLevine1993a,RajanZingales1998}, others hold that financial sector removes scarce resources from the economy \parencite{bolton2016cream,Tobin1984,axelson2015wall} and underpins greater exposure and vulnerability to crises, severely burdening the real sector in during periods of instability \parencite{Minsky1991,Stiglitz2000}. More recent papers also point towards decreasing returns to financial development and finance having negative consequences for growth when above a certain threshold \parencite{Arcandetal2012,LawSingh2014,RousseauWachtel2011}. 
 
We depart from the literature in two main features. First, we apply \ac{BMA} to solve the model uncertainty problem in growth regressions. The variety of theories of economic growth suggests a large number of determinants and results in considerable uncertainty about the ``true'' growth model. Using the \ac{BMA}, we can evaluate numerous regressors potentially relevant for economic growth and estimate their \ac{PIP}, the probability that they are relevant in explaining the dependent variable, additionally to the weighted mean and variance of the respective coefficients. \ac{BMA} essentially estimates varying combinations of explanatory variables and weights the coefficients using model fit. The methodology is solidly rooted in the statistical theory \parencite{Rafteryetal1997,Koopetal2007} and indirectly also helps us to tackle the potential of omitted variables bias, from which empirical work on finance and growth typically abstracts.

Second, we augment previous research by examining several financial development indicators simultaneously to account for the multidimensionality of financial systems. By jointly examining whether depth, stability, or efficiency is relevant for long-term growth, we re-examine and unify previous literature. The established functions of finance are difficult to operationalize in empirical research \parencite{Valickovaetal2014}, and there is no consensus on the measurement of financial development \parencite{KingLevine1993a}. The research dominantly uses depth of financial markets (credit / GDP ratio or stock market capitalization / GDP) as a measure of financial development. Employing the \ac{GFDD} and the indicators provided therein, we can approximate the function of the financial system in much more detail. We can discriminate between banking and stock markets as well as evaluate the relative importance of depth versus the alternative proxies of efficiency, stability, and access to finance. We may also reflect the claims that excessive financial development and financial instability are harmful to growth. Even though the data coverage is still somewhat limited, we contribute to the literature by considering these additional dimensions of the financial sector in our regression analysis to provide a more exact picture of finance-growth nexus. We complement the data on financial development by the long-term growth dataset of \textcite{Fernandezetal2001}, which provides a rich set of possible explanatory variables capturing various economic, political, geographical, and institutional factors.

We find that efficiency of financial intermediation is the only indicator of financial development, which is robustly related to economic growth and consistently shows very high \ac{PIP}. This result is consistent with the theoretical predictions sketched out by \textcite{Pagano1993}, who shows how the increased efficiency of financial intermediaries affects the channel between savings and investment and therefore leads to higher real growth. On the other hand, the relevance of the traditionally employed variables, such as credit to the private sector or stock market capitalization, is weaker. Additionally, we find no evidence for a non-linear effect of financial development. We subject our results to further robustness checks by focusing on different sample periods, employing alternative priors, and basic techniques to address endogeneity with no substantial effect on our conclusions. The policy implications of the results highlight the essential importance of measuring financial development to precisely describe its consequences. The regulatory changes in the financial industry should appreciate the relevance of financial intermediaries for long-term growth. We published the paper in \emph{The World Bank Economic Review}.

\emph{Chapter 3 - Finance and Wealth Inequality} extends the idea of distinct features of financial systems to the distribution of wealth. Wealth inequality markedly varies across countries \parencite{daviesetal2017}, and the interest of the paper is to uncover the drivers of these differences. Is it different degrees of redistribution, financial development, globalization, technological progress, education, economic development, or something else? Although measurement of wealth inequality advanced significantly \parencite{daviesetal2017,SaezZucman2016}, there is a lack of systematic evidence about the determinants of wealth inequality across countries.

The theoretical predictions of the wealth inequality drivers vary. Much discussed $r > g$ concept presented by \textcite{piketty2014} suggests there is a natural tendency towards increasing wealth inequality unless exogenously amended by redistribution or wars. The framework is criticized on many fronts, though \parencite{mankiw2015yes,blume2015capital,king2017literature}, and the cornerstone remains with distinct applications of dynamic quantitative models. The models\footnote{\textcite{DENARDI2017280} provide thorough overview of the model implementations.} critically rely on the saving motives of the individuals, and this leads us to the hypothesis of financial development being crucially relevant for wealth distribution. Another prediction about the financial system and wealth inequality arises from \textcite{pastor2016income} in which inequality is driven, among other things, by the ability of entrepreneurs to diversify their idiosyncratic risk. The empirical evidence is scarce as the research papers on inequality mostly turn to the distribution of income due to better data availability. Nonetheless, wealth is much more unevenly distributed than income \textcite{zucman2019,oecd2013crisis}, and income distribution is mostly used as an approximation of wealth distribution, while the latter would be more fitting given the theory \parencite{bagchi2015does}.

The lack of encompassing theoretical framework informs our methodological framework similarly as in the paper on finance and growth. We rely on \ac{BMA} in estimations to identify relevant determinants of wealth distribution. Moreover, we extend the analysis to address potential endogeneity more rigorously using the \ac{IVBMA}. \ac{IVBMA} mostly resembles the two-stage frequentist methods but accounts for model uncertainty in both stages. We further refresh and expand the set of regressors by constructing our original database, although conceptually, we similarly select the variables capturing economic, financial, institutional, regulatory, and political features of considered countries. We prefer the freedom about the choice of regressors over the comparability of our results with existing research as the paper is a pioneering work in this field. We also importantly update the indicators of financial development we employ. Rather than relying on single indicators capturing different dimensions of financial systems, we use complex indicators constructed from the \ac{GFDD} that describe the characteristics of financial systems by extracting information from multiple indicators in each dimension through principal component analysis.

We find that the set of key determinants is small, and financial development exerts a complex effect on wealth inequality. Whereas countries with deeper financial systems (large financial markets and financial institutions) exhibit greater wealth inequality, more efficient intermediation and access to finance are associated with less wealth inequality. Our results thus support the idea that sound financial systems may contribute to lower wealth inequality. Alongside financial development, we discover that education, redistribution, globalization, and political instability affect wealth distributions within countries. Better educated societies and higher redistribution of income support the more egalitarian distribution of wealth, while globalization and political instability increase wealth inequality. The conclusions offer apparent policy alternatives of countermeasures to increasingly unequal distributions of wealth in inclusive and efficient financial systems alongside better education. The paper is forthcoming in the \emph{Journal of International Money and Finance}.

\emph{Chapter 4 - Finance and Inequality - panel BMA approach} is the last follow-up in the series of papers on financial development. The theoretical predictions and their ambiguity resemble the ones on wealth inequality and finance. In contrast with wealth, income inequality and the relation to finance are subject to research much more frequently, but with conflicting outcomes. A fundamental divide appears between financial development on the extensive and intensive margin. On the extensive margin, it might lead to more equal opportunities and outcomes as access to credit by previously disadvantaged groups allows human capital accumulation \parencite{braunetal2019,galormoav2004} and formation of new firms \parencite{banerjeenewman1990,evans1989estimated}. On the contrary, the intensive margin of financial development might inordinately benefit the rich incumbents who leverage financial services for their further benefit or to protect their existing rents \parencite{GreenwoodJovanovic1990,perotti2007investor}.

The paper re-examines the literature on finance and inequality by applying panel \ac{BMA} techniques, once more identify the main determinants of income distribution within countries. I contribute to existing research by showing that: 1) finance has a significant role in shaping the distribution of income, 2) the complexity of the relationship arises from the characteristics of financial systems, and 3) the effect varies across different parts of the income distribution. 

Reflecting the conclusions of the preceding chapter, efficiency and access to financial institutions appears to have the inequality reducing role. The depth of the financial system seemingly does not influence the overall measure of the income distribution (Gini index). However, when the focus is on the top income shares, the size of the financial markets and institutions coincide with a more concentrated distribution of income. Additionally, the paper also provides evidence on other popular hypotheses about increasing income inequality exploring the education \parencite{goldin2009race}, globalization \parencite{Jaumotte2013}, or technological progress \parencite{dabla2015causes}. Interestingly, the results associate globalization proxied by the trade openness with a higher concentration of income at the top of the distribution, but its relevance diminishes when the measure of inequality is the Gini index. A higher level of education index tends to mitigate overall income inequality but remains irrelevant for the concentration of income among the top 1\%. For the technological progress, indirect evidence using the investment into research and development and intellectual property suggests a positive relationship, supporting the idea of increasing inequality due to advancement in technology.

Alongside the income Gini index, the chapter employs inequality measures that have been recently scrutinized in the literature. Tax data used to compile the top income shares since \textcite{piketty2003income} is deemed superior to the survey data. However, it has its limitations as a significant part (up to 40\% in the exemplary case of the US) of personal and national income is missing in the tax returns. It could be due to the government deliberately leaving some income untaxed (pension and other insurance benefits) or tax evasion by the tax filers \parencite{johns2010distribution, alstadsaeter2019tax}. The studies attempt to assign this portion of missing income among the units used in the computation and vary in their methodological approaches. Additional differences may arise due to considered units of observation themselves (tax unit vs. individuals) and dynamically changing demographic structure. The resulting variation in the suggested concentration of income ranges from 2 to 12 percentage points across the studies \parencite{auten2019top}. If we want to set up the right policies to address inequality trends, I believe convergence in this issue is fundamental to draft appropriately scaled policies. Simultaneously, the differences do not disqualify the search for the channels through which the future policies could be targeted and executed. As long as the measurement issues are not dramatically heterogeneous across countries and time, consistently collected and constructed data may inform us of the causes and consequences of inequality irrespective of the precise numbers put on various measures of inequality.

To summarize the dissertation's policy implications, I want to emphasize the importance of the financial system's functions in affecting long-term economic growth and the dynamics of income and wealth distributions. Above all, the efficiency of financial intermediation and inclusiveness of finance appear as significant determinants. Efficiency seems particularly vital for growth, while access to financial services associates with more equal distributional outcomes. Regulatory reforms often overlook the impact of the policies, especially in the case of inequalities. The policy-makers should put a substantial effort into impact assessment capabilities. In terms of efficiency, the efforts could concentrate on providing a competitive financial environment and support better allocation of savings through the financial industry. These may take forms of allowing foreign entities' entry, careful support of microcredit institutions with lending aimed at new business opportunities, and establishing the ground for new financial products expanding the real economic opportunities rather than leverage. Our estimates also show that high -- 'excessive' -- levels of financial depth lead to concentration of wealth and income at the top of the distribution. To a large extent, the question about the desirability of such an outcome is normative, but if the policy-makers decide to take action, the off-setting channels arise in expanding opportunities through education, financial inclusion, and mindful regulation.

% Relying on the global sample of countries. We identify many potentially relevant regressors hypothesized in the literature as potential drivers of wealth distribution. 

% Financial development debate following the crisis\dots

% \ac{BMA} useful in examining uncertain relationships\dots + what is it, well grounded in theory

% Roles of finance\dots + multidimensionality

\newpage
\printbibliography[heading=subbibliography]
\addcontentsline{toc}{section}{References}
\end{refsection}