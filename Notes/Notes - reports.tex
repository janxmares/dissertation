%!TEX program = pdflatex
%!BIB program = biber

\documentclass[a4paper,11pt]{article}
\usepackage{graphicx}
% \usepackage{fontspec}
% \usepackage[round]{natbib} %standard package for bibliography
\usepackage{hyperref}
\usepackage{parskip}

\usepackage[style=apa]{biblatex}
\DeclareSourcemap{
  \maps[datatype=bibtex]{
    \map{ % set fields in .bib file to ignore
      \step[fieldset=issn, null]
      \step[fieldset=isbn, null]
      \step[fieldset=abstract, null]
      \step[fieldset=doi, null]
      \step[fieldset=url, null]
    }
  }
}
%
\addbibresource{literature.bib}	% bib file

%
%
%
%
%

\begin{document}

Instruments in \textcite{edisonetal2002} are legal origin and absolute latitude (distance from the equator).

\textbf{\textcite{Tobin1984}} distinguishes different approaches to efficiency. First, he talks about \textit{information-arbitrage} efficiency, i.e. it is on average impossible to gain on trading using publicly available information\footnote{Anecdote with the bill on the sidewalk, if it was really there, it would have been picked up.}. Second, the market is efficient if its valuations reflect accurately the future payments of the assets their holders are entitled to receive - \textit{fundamental valuation efficiency}. Third, efficiency reflects the ability of the markets to allow economic agents to insure the deliveries of goods and services in all future contingencies, either by surrendering some of their own resources now or by contracting to deliver them in specific future contingencies - \textit{full-insurance efficiency.}. Last, the functional efficiency reflect the economic functions of the financial industry. That is, in Tobin's view, pooling  of risks and their allocation to those most able and willing to bear them, generalized insurance function as mentioned previously, facilitation of transactions by providing mechanisms and networks of payments, the mobilization of saving for investment in physical and human capital (domestic and foreign, private and public, allocation of saving to their more socially productive uses), and ``gambling per se''.

Ad 1) The prices react to the news promptly and fully - and conceivably with little or no trading.

Ad 2) markets move up and down much more than what could be accounted to changes in fundamentals or the rates at which they are discounted. Shiller documents it in several of his papers, also for the bond markets.
JM Keynes and the pretty face competition (decided on what the other will like the best), taken to another (Tobin says 3rd) level. ``There is no clear evidence from experience that the investment policy which is \textit{socially advantageous} coincides with that which is most profitable''. Suggestion of transfer tax. The liquidity might be unnecessarily high (illiquidity might be worse, but he was writing during Great Depression). ``The spectacle... has sometimes moved me towards the conclusion that to make the purchase of an investment permanent and indissoluble, like marriage, (sic!), except by reason of death or other grave cause, might be a useful remedy...''.

Ad 3) Markets require resources to operate, therefore, having a complete set of (often ridiculous) options would not be efficient. Many useful instruments exist in the markets, but there are niches, which could potentially be beneficial: flexible mortgages for young families, assets which would allow elderly to consume from they wealth (e.g., residential ownership), price-indexed instruments (we have covered that since!) - based on adjusted CPI, which abstracts from prices of oil, imports, exchange-rate  movements, and indirect taxes). The new future contracts do no stretch very far into the future. They mainly serve to allow greater leverage to short-term speculators and arbitrageurs, and to limit losses in one direction or the other. In approving the instruments, the authorities should bear in mind whether they really fill the gaps in the menu and enlarge the opportunities for Arrow-Debreu insurance, not just enlarge opportunities for speculation and financial arbitrage. 

Ad 4) Very few new issuance serve for new investment (which would by large be possible from retained earnings). There is some indirect affect of financial markets abstracted from - some firms fund the others (the previous statement is based on aggregate numbers), but very little real investment is actually funded directly by markets. Banking shows symptoms of monopolistic competition - product differentiation, large spending on advertising, pricing base on leadership of large firms - the prime rate. Proliferation of non-standardized products is costly, beyond a certain point it is not necessarily a service to the consuming public. Move from monopolistic competition to potential oligopoly, also local banker knew its borrower the best, some loans will not be given by a foreign-owned bank.

Intermediation between surplus and deficit households is a great service to the economy. 

In conclusion, unbecoming an academic, ``we are throwing more and more of our resources, including the cream of our youth, into financial activities remote from the production of goods and services, the activities that generate high private rewards disproportionate to their social productivity.'' Potentially negative sum game for general public. The industry attracts short-horizon speculators and middlemen, and distorts or dilutes and influence of fundamentals on prices. \textbf{We should provide greater deterrents to transient holdings of financial instruments and larger rewards for long-term investors.}

\textbf{Questions - Geršl}

Do you mean the \href{https://stats.bis.org/statx/srs/table/f2.1}{total credit from BIS} . Assuming total credit instead of bank credit does not alter the results.

Some studies have explored ways to incorporate the missing income. But this has resulted in widely divergent results. Such differences highlight the sensitivity of results to choices about which missing income sources to include, about the unit of observation, and about how missing income is allocated. Thus, the use of tax return data can lead to distorted estimates of inequality trends if researchers are not careful. 

Tax returns miss about 40 percent of personal and national income in recent years, including underreported income and employer-provided insurance benefits. Tax reforms, especially the Tax Reform Act of 1986, have significantly changed the rules and incentives for realizing and reporting income. Using tax units as the unit of observation, as in PS, also has limitations. Marriage rates have declined over time, except for those at the top of the income distribution. As a result, basing income groups on tax units rather than adults or all individuals mechanically increases top income shares.

This paper examined the sensitivity of estimated top income shares to researchers’ choices about which sources of income to include, how to allocate income missing from tax returns, and how to account for changes in family structure and changes in tax laws. Our analysis shows that accounting for these factors results in significantly lower levels of and smaller increases in estimated top income shares (Auten \& Splinter, 2019).

Economist - the numbers vary, precision is a question. Caution should be taken when considering specific policies. There exist some other than 'wealth tax' to tackle the current issues. Tackling excessive market power of some companies, soaring housing prices.

Conventional thoughts - first, income and wealth shares of top 1\% have soared in last four to five decades. Second, the incomes of middle-earners stagnated. Third, real-wages have barely risen although the productivity done so. As a consequence, increasing share of the GDP goes to investors int he form of interest, dividends, and capital gains, rather than to labour through wages. Fouth, reinvestment means that wealth inequality has risen too.

Two-fifths of national income does not show on tax returns. 1) pension system retirement savings, 2) tax evasion (Andrew Johns, Joel Slemrod). Pass-through firms.

Labour share of income did not fall in advances economies, except for the US when adjusted for self-employment and property income.

Pension funds own 50\% of US stocks as opposed to 4\% in 1960.

On endogeneity and reverse causality:

McKinnon–Shaw models, which highlight the role of financial development in the process of economic growth, the endogenous financial development and growth models show reciprocal interactions between these two variables. That is, on the one hand, a higher level of economic development stimulates more demand for financial services, leading to increased competition and efficiency in the financial intermediaries and financial markets. On the other hand, the provision of timely and valuable information by financial intermediaries to investors allows investment projects to be launched more efficiently, and this enhances capital accumulation and economic growth \parencite{Ang2008}. - WORD BY WORD

In particular, the class of endogenous growth models points towards the reciprocal relation between growth and financial development. In these models, the demand for financial services grows with an increased level of economic development. The higher demand for financial services boosts competition and improves the efficiency of financial intermediation. Simultaneously, the finance's better efficiency enhances the screening and capital allocation of investment, accelerating capital accumulation, and consequently, economic growth.

While the economy may not recover to the potential of pre-crisis levels of economic output (Cerra \& Saxena, 2008).

A country that has 1 percentage point higher GDP growth in the year before the crisis has a shorter and shallower contraction (by one quarter and 0.5 per cent of GDP, respectively). This result confirms our belief that recession-induced systemic crises have higher output costs than those crises beginning when the economy is growing at a relatively high rate (Cecchetti et al., 2009).

By altering attitudes towards risk, as well as increasing the level of government debt and the size of central banks’ balance sheets, systemic crises have the potential to raise real and nominal interest rates and consequently depress investment and lower the productive capacity of the economy in the long run. We looked for evidence of these effects and found
that a number of crises had lasting, negative impacts on GDP. In some countries this was a result of an immediate, crisis-induced drop in the level of real output combined with a permanent decline in trend growth. In other cases, we find that the growth trend increased following the crisis but that the immediate drop was severe enough that it took years for the
economy to make up for the crisis-related output loss (Cecchetti et al., 2009). 

NIM measure comment
Thank you, your observation is indeed correct. In comparison with the other chapter, we may have oversimplified the matter

Endogeneity

Three main reasons for endogeneity: measurement error, omitted variable bias, \textbf{reverse causality}.

Thank you for this observation. Although the issue of endogeneity gets mentioned throughout the paper, it might come out as underappreciated. I have explicitly added a part on potential endogeneity and the limitations of our attempts to address it. While we efficiently deal with the potential omitted variable bias by applying BMA, the risk lies in the reverse causality. 

In addition, I have added the results on the two-stage least squares BMA that was previously only mentioned in one of the footnotes and results on request. The results support the baseline finding of the efficiency of intermediation being conducive to growth.

Genetic distance might not be the best instrument in the case of economic growth. The risk of violating exclusion restriction is high as it genetic distance most likely affects also other institutions other than financial. This could be, for example, general trust, rule of law, or spread of other than financial technologies.

To control for endogeneity, we include lagged values of the explanatory variables in the regressions. We do not use second and higher lags to avoid autocorrelation with the current error term. Table 2 reports the results. Our findings of the effect of financial development on income inequality are robust (Bittencourt et al., 2019).


\textbf{Efficiency}

In the perfect world with no assymetric information and transactions costs, no interest rate spreads between saving and borrowing exist.

\textbf{Before and after-tax Gini}


\textbf{Bizzare}

\href{https://dspace.tul.cz/bitstream/handle/15240/26635/EM_3_2018_10.pdf?sequence=1&isAllowed=y}{DEA evidence on efficiency financial intermediation for Slovakia}

on the top of that \href{https://dspace.tul.cz/bitstream/handle/15240/21377/EM_4_2017_03.pdf?sequence=1&isAllowed=y}{this}

Cernohorsky on using the cointegration for determining what type of loans matter for growth in the Czech Republic.

\printbibliography

\end{document}