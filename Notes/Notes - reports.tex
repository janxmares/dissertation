%!TEX program = pdflatex
%!BIB program = biber

\documentclass[a4paper,11pt]{article}
\usepackage{graphicx}
% \usepackage{fontspec}
% \usepackage[round]{natbib} %standard package for bibliography

\usepackage[style=apa]{biblatex}
\DeclareSourcemap{
  \maps[datatype=bibtex]{
    \map{ % set fields in .bib file to ignore
      \step[fieldset=issn, null]
      \step[fieldset=isbn, null]
      \step[fieldset=abstract, null]
      \step[fieldset=doi, null]
      \step[fieldset=url, null]
    }
  }
}
%
\addbibresource{literature.bib}	% bib file

%
%
%
%
%

\begin{document}

Instruments in \textcite{edisonetal2002} are legal origin and absolute latitude (distance from the equator).

\textbf{\textcite{Tobin1984}} distinguishes different approaches to efficiency. First, he talks about \textit{information-arbitrage} efficiency, i.e. it is on average impossible to gain on trading using publicly available information\footnote{Anecdote with the bill on the sidewalk, if it was really there, it would have been picked up.}. Second, the market is efficient if its valuations reflect accurately the future payments of the assets their holders are entitled to receive - \textit{fundamental valuation efficiency}. Third, efficiency reflects the ability of the markets to allow economic agents to insure the deliveries of goods and services in all future contingencies, either by surrendering some of their own resources now or by contracting to deliver them in specific future contingencies - \textit{full-insurance efficiency.}. Last, the functional efficiency reflect the economic functions of the financial industry. That is, in Tobin's view, pooling  of risks and their allocation to those most able and willing to bear them, generalized insurance function as mentioned previously, facilitation of transactions by providing mechanisms and networks of payments, the mobilization of saving for investment in physical and human capital (domestic and foreign, private and public, allocation of saving to their more socially productive uses), and ``gambling per se''.

Ad 1) The prices react to the news promptly and fully - and conceivably with little or no trading.

Ad 2) markets move up and down much more than what could be accounted to changes in fundamentals or the rates at which they are discounted. Shiller documents it in several of his papers, also for the bond markets.
JM Keynes and the pretty face competition (decided on what the other will like the best), taken to another (Tobin says 3rd) level. ``There is no clear evidence from experience that the investment policy which is \textit{socially advantageous} coincides with that which is most profitable''. Suggestion of transfer tax. The liquidity might be unnecessarily high (illiquidity might be worse, but he was writing during Great Depression). ``The spectacle... has sometimes moved me towards the conclusion that to make the purchase of an investment permanent and indissoluble, like marriage, (sic!), except by reason of death or other grave cause, might be a useful remedy...''.

Ad 3) Markets require resources to operate, therefore, having a complete set of (often ridiculous) options would not be efficient. Many useful instruments exist in the markets, but there are niches, which could potentially be beneficial: flexible mortgages for young families, assets which would allow elderly to consume from they wealth (e.g., residential ownership), price-indexed instruments (we have covered that since!) - based on adjusted CPI, which abstracts from prices of oil, imports, exchange-rate  movements, and indirect taxes). The new future contracts do no stretch very far into the future. They mainly serve to allow greater leverage to short-term speculators and arbitrageurs, and to limit losses in one direction or the other. In approving the instruments, the authorities should bear in mind whether they really fill the gaps in the menu and enlarge the opportunities for Arrow-Debreu insurance, not just enlarge opportunities for speculation and financial arbitrage. 

Ad 4) Very few new issuance serve for new investment (which would by large be possible from retained earnings). There is some indirect affect of financial markets abstracted from - some firms fund the others (the previous statement is based on aggregate numbers), but very little real investment is actually funded directly by markets. Banking shows symptoms of monopolistic competition - product differentiation, large spending on advertising, pricing base on leadership of large firms - the prime rate. Proliferation of non-standardized products is costly, beyond a certain point it is not necessarily a service to the consuming public. Move from monopolistic competition to potential oligopoly, also local banker knew its borrower the best, some loans will not be given by a foreign-owned bank.

Intermediation between surplus and deficit households is a great service to the economy. 

In conclusion, unbecoming an academic, ``we are throwing more and more of our resources, including the cream of our youth, into financial activities remote from the production of goods and services, the activities that generate high private rewards disproportionate to their social productivity.'' Potentially negative sum game for general public. The industry attracts short-horizon speculators and middlemen, and distorts or dilutes and influence of fundamentals on prices. \textbf{We should provide greater deterrents to transient holdings of financial instruments and larger rewards for long-term investors.}

\textbf{Questions - Geršl}
Do you mean the total credit from BIS: https://stats.bis.org/statx/srs/table/f2.1. Assuming total credit instead of bank credit does not alter the results.

\printbibliography

\end{document}